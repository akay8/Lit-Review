\documentclass{article}
\usepackage{amsmath}

\usepackage[backend=biber,style=numeric,sorting=none]{biblatex}
\addbibresource{First_Year_Reading.bib}

\usepackage{hyperref} % Hyperlinks
\hypersetup{
    colorlinks=true,
    linkcolor=blue,
    filecolor=blue,      
    urlcolor=blue,
    citecolor=blue
    }

\usepackage{ragged2e} % to stop linebreak errors

\begin{document}
\title{References Reference}
\author{Abi Kaye}
\maketitle

% \loadspellchecklist[en][wordlist.txt]
% \setupspellchecking[state=start]

\section{General Reads}

\raggedright

This is a test document. Let's test if this compiles correctly: 

\bigskip

This is a potential way of reducing light shift in compact clock systems \cite{2024calossoLaserfrequencyStabilizationUsingLight} 

\smallskip

This is the first use of vapour cells in atomic clocks / frequency references \cite{2002kitchingMiniatureVaporcellAtomicfrequencyReferences} 

\smallskip

A recent review article of 2 photon Rb compact clocks \cite{2025obaze-adelekeComprehensiveReviewRubidiumTwoPhoton} 

\smallskip

Sean's thesis - good vapour cell and general reference \cite{2024dyerDevelopmentMicrofabricatedVapourCell} 

\smallskip

French thesis - sean recommends - looks at LCVR and the noise it adds to system \cite{2017abdelhafizDevelopmentMetrologicalCharacterizationHighperformance} 

\smallskip

Rachel Cannon's thesis - good explanation of error signals / lock in detection \cite{2024cannonMiniaturisedHighreliabilityLasersQuantum} 

\smallskip

New paper on short term stability of 87Rb 2 photon clock, nice diagrams \cite{2025callejoShorttermStabilityMicrocellOptical}

\smallskip

Eilidh's journal club paper - 776nm fluorescence detection \cite{2024beardTwophotonRubidiumClockDetecting} 

\smallskip

Paper by Aidan and Rachel O on how to characterise noise in an ECDL \cite{2021daffurnSimplePowerfulDiodeLaser}

\smallskip

Aidan suggestion 2 - Doppler thermometry and how to fit spectra nicely \cite{2025agnewPracticalPrimaryThermometryAlkalimetalvapour}

\smallskip

Steck 87Rb \cite{AlkaliLineData}

\smallskip

More interest than anything - a new Python package atomSmltr for simulation laser cooling and MOTs \cite{2025weillAtomSmltrModularPythonPackage}

\smallskip

The original 3 cornered hat maths - first paper but not useful reading \cite{1948grubbsEstimatingPrecisionMeasuringInstruments}

\smallskip

A general review article on metrology - should read all through. The first mention of a three cornered hat and a good description including the maths \cite{1999levineIntroductionTimeFrequencyMetrology}

\smallskip

Enrico Rubiola's phase noise / frequency noise chart - read! \cite{2023rubiolaEnricosChartPhaseNoise} \cite{2023rubiolaCompanionEnricosChartPhase}

\section{Iodine Clocks}

The main paper for optical iodine clocks (Vector Atomic, Roslund et al) \cite{2024roslundOpticalClocksSea} and their recent conference proceeding \cite{2024roslundMolecularIodineOpticalAtomic}

\smallskip

German group (Wust et al) who have made iodine frequency references for space \cite{2024wustOpticalIodineClocksFuture}

\smallskip

A thesis on iodine frequency references that has some nice explanations in it \cite{2016adsersenIodineBasedThermalOptical}

\smallskip

Original (?) paper on use of unsaturated iodine cell for frequency stability. Short paper, easy to read. 
Takeaways: coating inside of cell walls with iodine then adding known amount (moles) of gas produces linear temp vs 
pressure relationship, when operating temperature is 10 degrees or more (good for frequency stability). Could consider
higher vapour pressure / moles of gas for better SNR but need to operate at higher temps to get linear region. \cite{1993quinnNewTypeIodineCell}

\smallskip

More recent paper on unsaturated iodine cells. Uses modulated transfer spectroscopy (MTS), where probe beam in SAS is modulated using an EOM
to provide a locking signal. uses a 25cm length glass cell!! and tonnes of optics we wont be able to use. ~10$^{-15}$ at 10,000 s though,
which is impressive \cite{2024zhangHighPerformanceMolecularIodine}

\smallskip 

A better explanation / example of MTS (see above iodine spectroscopy) \cite{2024meloLaserFrequencyStabilizationModulation}

\section{Pill Characterisation}

Thesis sections 4.6 onwards \cite{2017aldousEnablingTechnologiesIntegratedAtom}

\smallskip

Other thesis \cite{2018dragomirColdAtomsYourPocket}

\smallskip

\printbibliography

\end{document}