\documentclass{article}
\usepackage{amsmath}

\usepackage[backend=biber,style=numeric,sorting=none]{biblatex}
\addbibresource{First_Year_Reading.bib}

\usepackage{hyperref} % Hyperlinks
\hypersetup{
    colorlinks=true,
    linkcolor=blue,
    filecolor=blue,      
    urlcolor=blue,
    citecolor=blue
    }

\usepackage{ragged2e} % to stop linebreak errors

\begin{document}
\title{References Reference}
\author{Abi Kaye}
\maketitle

\section{Introduction}

\raggedright

This is a test document. Let's test if this compiles correctly: 

This is a potential way of reducing light shift in compact clock systems \cite{2024calossoLaserfrequencyStabilizationUsingLight} 

This is the first use of vapour cells in atomic clocks / requency references \cite{2002kitchingMiniatureVaporcellAtomicfrequencyReferences} 

A recent review article of 2 photon Rb coompact clocks \cite{2025obaze-adelekeComprehensiveReviewRubidiumTwoPhoton} 

Sean's thesis - good vapour cell and general reference \cite{2024dyerDevelopmentMicrofabricatedVapourCell} 

French thesis - sean recommends - looks at LCVR and the noise it adds to system \cite{2017abdelhafizDevelopmentMetrologicalCharacterizationHighperformance} 

Rachel Cannon's thesis - good explaination of error signals / lock in detection \cite{2024cannonMiniaturisedHighreliabilityLasersQuantum} 

New paper on short term stability of 87Rb 2 photon clock, nice diagrams \cite{2025callejoShorttermStabilityMicrocellOptical}

Eilidh's journal club paper - 776nm flourescence detection \cite{2024beardTwophotonRubidiumClockDetecting} 

Steck 87Rb \cite{AlkaliLineData}



\printbibliography

\end{document}